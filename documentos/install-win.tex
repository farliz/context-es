\setuppapersize[A4, landscape]
\mainlanguage[es]
\setuplayout[
		header=1.5cm,
		footer=1.5cm,
		height=fit,
		topspace=0.5cm,
		width=fit,
		backspace=1.5cm,
		rightmargin=0.5cm,
		leftmargin=0.5cm,
		leftmargindistance=0pt,
		rightmargindistance=0pt,
		grid=yes, columndistance=12pt,
		columns=3
		]

\setupheader[
		state=on,
		strut=yes,
		style=\tfx\bf\ss,
		]
\setupheadertexts[Universidad Técnica de Manabí][]

\setupfooter[
		state=on,
		strut=yes,
		style=\tfx\bf\ss,
		]
\setupfootertexts[Made with \CONTEXT][http://wiki.contextgarden.net]

\setuphead[title][
		style=\tfc\bf\ss,
		align=middle,
		]
\setuphead[subject][
		style=\tfa\bf\ss,
		after=\blank,
		]
\setupbodyfont[lbr, 18pt]
\definefont[ctex][name:LinLibertine_RB_G*default at 30pt]
\definefont[firait][name:FiraSans-Italic*default at 18pt]
\definefont[fira][name:FiraSans-Regular*default at 18pt]
%\setupindenting[yes, big]
\definehighlight[men][color=blue]
\setuppagenumber[state=stop]

\setupalign[right]
\setupcolors[state=start]
\setupinteraction[state=start, color=blue]

%\definelayer[logo][width=.98\paperwidth,height=.97\paperheight,state=repeat]
%
%\setupbackgrounds[page][background={foreground,logo}]
%
%\setlayer
        %[logo]
        %[preset=righttop,
         %hoffset=1cm]
        %{\externalfigure[figures/utm_logo][height=4cm,3.5cm]}



\starttext
\startalignment [center]
\title{Guía de instalación de \CONTEXT ~\&~TeXworks para Windows (cualquier versión)}
\stopalignment
\blank

% INSTALAR CONTEXT

\subject{Primer paso:  instalar \CONTEXT}\\

\startalignment
\startitemize[n, packed]

\item Descargar el archivo ZIP del siguiente enlace \goto{http://minimals.contextgarden.net/setup/context-setup-mswin.zip}[url(http://minimals.contextgarden.net/setup/context-setup-mswin.zip)].

\item Guardar este archivo en el Disco local C: y extraer en este directorio. Aparecerá una carpeta de nombre {\ss{context}}, en la cual se hallará un archivo de nombre {\ss{first-setup.bat}}.
\item Ejecutar como administrador el archivo {\ss{first-setup.bat}}. Se abrirá una
  terminal de Windows que mostrará el proceso de descarga de los  archivos. 

\item Esperar a que la ventana se cierre automáticamente. Este proceso puede tardar según la velocidad de conexión a internet.

\stopitemize
\stopalignment

%INSTALAR TEXWORKS

\subject{Segundo paso: instalar TeXworks}\\

\startalignment
\startitemize[n, packed]
\item Descargar el archivo ejecutable desde la web oficial: \goto{http://www.tug.org/texworks/}[url(http://www.tug.org/texworks/)]

\item Ejecutar el archivo e instalar con los valores por defecto.

\item Ya instalado, ejecutar el programa {\ss{TeXworks}}, dirigirse al menú  {\ss{Editar}} y a la opción {\ss{Preferencias}}.

\item Una vez abierta la ventana de {\ss{Preferencias}}, dirigirse a la pestaña {\ss{Compilación}} y en esta a la sección {\ss{Ubicación de TeX y programas relativos}}, en la que se encuentra un botón con el signo {\ss\bf{+}}, sobre el cual hay que hacer click.

\item Se abrirá una nueva ventana, en esta se realizará la ruta a continuación:\\ {\ss{Disco local C: ---> context ---> tex ---> texmf-mswin ---> bin}}.\\ Luego, dar click en el botón {\ss{seleccionar carpeta}}.

\item Una vez que aparezca el directorio {\ss{C:/context/tex/texmf-mswin/bin}} en la lista, llevarlo hasta la {\bf{primera}} posición utilizando el botón con la flecha verde hacia arriba.

\item En la misma ventana de {\ss{Preferencias}}, en la sección {\ss{Herramientas para procesamiento}} seleccionar como predeterminada la opción {\ss{ConTeXt (LuaTeX)}}. El resultado hasta este punto debe coincidir con lo mostrado en la Figura 1 (\in{Fig.}[fig:f1]). Dar click en el botón {\ss{OK}}

\stopitemize
\stopalignment 

\placefigure
[]
[fig:f1]
{Correcta configuración de las preferencias de TeXworks}
{\externalfigure[../talleres/figuras/setup][width=.67\textwidth]}

%ATAJOS DEL TECLADO

\subject{Tercer paso: atajos del teclado}

\startalignment
\startitemize[n,packed]

\item Entrar a la siguiente dirección web:

  \goto{https://github.com/farliz/context-ec/blob/gh-pages/documentos/tw-context.txt}[url(https://github.com/farliz/texworks-context/blob/master/tw-context.txt)]. 

\item Crear un archivo de texto en el escritorio con el nombre de {\ss{tw-context}} y pegar el código expuesto en la página web en este archivo.

\item Dirigirse al menú {\ss{Scripts}} de TeXworks, luego a la opción {\ss{Scripts de TeXworks}} y luego a la opción {\ss{Carpeta de Scripts}}. Se abrirá una ventana, dentro de la cual hay que regresar un directorio, el cual contendrá la carpeta {\ss{completions}} donde yace un archivo de texto de nombre {\ss{tw-context}}.

\item Borrar este archivo de la carpeta y en su lugar pegar el archivo que creamos en el escritorio.

\stopitemize
\stopalignment

 
\stoptext