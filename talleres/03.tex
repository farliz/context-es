\usemodule[pre-01]
\mainlanguage[es]
\usemodule[chart]
%\showframe
\setupcolors[state=start]
\definelayer[LogoUTM]
\setupbodyfont[cmr,13pt]
\setuptype[color=blue]
\setuphead[subject][
		style=\tfa\bf
		]
\setuplayout[
		backspace=0.5cm,
		width=17.5cm,
		textwidth=16.25cm,
		]
\setupinteraction[
		state=start,
		color=middlecyan,
		style=\bf\ss
		]

\starttext
\StartTitlePage
  \CONTEXT ~~\&~~TeXworks\\  
{\tfxx
\goto{http://wiki.contextgarden.net}[url(http://wiki.contextgarden.net)]}
\blank
Taller 03
\blank
Tabulación
\blank[2cm]
\CONTEXT -ec\par   
{\externalfigure[figures/utm_logo][width=2.5cm]}
\StopTitlePage 

\Topics{Contenido}

%%%%%%%%%%%%%%%%%%%%%%%%%%%%%%%%%%%%%%%%%%%%%%%%%%%%%%%%%%%%%%%%%%%%%%%%%%%%
\Topic[tab]{Tabulación}

{
\color[blue]

{\tt \{}\\
\type{\bTABLE}\\
{\tt Contenido de la tabla}\\
\type{\eTABLE}\\
{\tt \}}
}

\blank

Comandos del contenido de la tabla:

\startitemize[packed]
	\item {\bf \type{\bTR}}: inicio de fila.
	\item {\bf \type{\eTR}}: fin de fila.
	\item {\bf \type{\bTD}}: inicio de celda.
	\item {\bf \type{\eTD}}: fin de celda.
\stopitemize

\page

Ejemplo:

\blank

{\tt \color[blue]{\{}}\\
\type{\bTABLE}\\
\type{\bTR \bTD Uno  \eTD \bTD Dos    \eTD \eTR}\\
\type{\bTR \bTD Tres \eTD \bTD Cuatro \eTD \eTR}\\
\type{\eTABLE}\\
{\tt \color[blue]{\}}}\\

\blank

Da como resultado:

\blank

{\bTABLE
\bTR \bTD Uno \eTD \bTD Dos  \eTD \eTR
\bTR \bTD Tres \eTD \bTD Cuatro \eTD \eTR
\eTABLE}

\blank

\startitemize
\item Las dimensiones de las celdas se ajustan automáticamente al contenido.

\page

\item Para definir dimensiones específicas se aplica la siguiente configuración:
\stopitemize

\blank

{\tt \color[blue]{\{}}\\
\type{\bTABLE}\\
\type{\bTR \bTD[width=2cm,height=2cm] Uno \eTD \bTD[width=4cm] Dos \eTD \eTR}\\
\type{\bTR \bTD[height=1cm] Tres \eTD \bTD Cuatro \eTD \eTR}\\
\type{\eTABLE}\\
{\tt \color[blue]{\}}}\\

\blank

Da como resultado:

\blank

{\bTABLE
\bTR\bTD[width=2cm,height=2cm] Uno \eTD\bTD[width=4cm] Dos \eTD\eTR
\bTR\bTD[height=1cm] Tres \eTD\bTD Cuatro \eTD\eTR
\eTABLE}

\page

En otros casos, es necesario modificar las características de toda una fila o columna. En ese caso se debe anteponer a la tabla el siguiente comando:

\blank

\type{\setupTABLE[id1][id2][ajuste=valor,ajuste=valor]}

\blank

Donde:
\startitemize
\item {\bf id1} representa la identificación de fila (row) o columna (column) en cuestión, 
\item {\bf id2} representa la identificación de la fila o columna en específico, ya sea por número (1,2,3...) o por paridad (odd o even).
\item {\bf Ajustes}: width, align, style, background, backgroundcolor, foregroundcolor, frame, topframe, bottomframe, entre otros.
\stopitemize


\page

Ejemplo:

\blank

{\tt \color[blue]{\{}}\\
\type{\setupTABLE[row][1][style=\bf,align=left]}\\
\type{\setupTABLE[column][2][background=color,backgroundcolor=yellow]}\\
\type{\setupTABLE[row][2][style=\ss,bottomframe=off]}\\
\type{\bTABLE}\\
\type{\bTR \bTD Uno  \eTD \bTD Dos    \eTD \eTR}\\
\type{\bTR \bTD Tres \eTD \bTD Cuatro \eTD \eTR}\\
\type{\eTABLE}\\
{\tt \color[blue]{\}}}\\

\blank

Da como resultado:

\blank
{
\setupTABLE[row][1][style=\bf,align=left]
\setupTABLE[row][2][style=\ss,bottomframe=off]
\setupTABLE[column][2][background=color,backgroundcolor=yellow]
{\bTABLE
\bTR \bTD Uno \eTD \bTD Dos  \eTD \eTR
\bTR \bTD Tres \eTD \bTD Cuatro \eTD \eTR
\eTABLE}
}

\Topic[eg]{Ejemplos de aplicación}

\type{\bTABLE}\\
\type{\bTR \bTD[nr=3] 1 \eTD \bTD[nc=2] 2/3 \eTD \bTD[nr=3] 4 \eTD \eTR}\\
\type{\bTR \bTD 2 \eTD \bTD 3 \eTD \eTR}\\
\type{\bTR \bTD 2 \eTD \bTD 3 \eTD \eTR}\\
\type{\bTR \bTD[nc=3] 1/2/3 \eTD \bTD 4 \eTD \eTR}\\
\type{\bTR \bTD 1 \eTD \bTD 2 \eTD \bTD 3 \eTD \bTD 4 \eTD \eTR}\\
\type{\eTABLE}\\

\blank

{\bTABLE
\bTR \bTD[nr=3] 1 \eTD \bTD 3 \eTD \eTR
\bTR \bTD 2 \eTD \bTD 3 \eTD \eTR
\bTR \bTD 2 \eTD \bTD 3 \eTD \eTR
\bTR \bTD[nc=3] 1 2 3 \eTD \bTD 4 \eTD \eTR
\bTR \bTD 1 \eTD \bTD 2 \eTD \bTD 3 \eTD \bTD 4 \eTD \eTR
\eTABLE}

\page

\type{\setupTABLE[row][odd] [background=color,backgroundcolor=red,}\\
\type{frame=off]}\\
\type{\setupTABLE[row][even][background=color,backgroundcolor=gray,}\\
\type{frame=off]}\\
\type{\bTABLE}\\
\type{\bTR \bTD first \eTD \bTD alpha \eTD \bTD one \eTD \eTR}\\
\type{\bTR \bTD second \eTD \bTD beta \eTD \bTD two \eTD \eTR}\\
\type{\bTR \bTD third \eTD \bTD gamma \eTD \bTD three \eTD \eTR}\\
\type{\eTABLE}\\

\blank

{
\setupTABLE[row][odd][
		background=color,
		backgroundcolor=red,
		frame=off
		]
\setupTABLE[row][even][
		background=color,
		backgroundcolor=gray,
		frame=off
		]
\bTABLE
\bTR \bTD first \eTD \bTD alpha \eTD \bTD one \eTD \eTR
\bTR \bTD second \eTD \bTD beta \eTD \bTD two \eTD \eTR
\bTR \bTD third \eTD \bTD gamma \eTD \bTD three \eTD \eTR
\eTABLE
}

\page

\type{\setupTABLE[frame=off]}\\
\type{\setupTABLE[column][first][leftframe=on]}\\
\type{\setupTABLE[column][last][rightframe=on]}\\
\type{\setupTABLE[row][first][topframe=on]}\\
\type{\setupTABLE[row][first,last][bottomframe=on]}\\
\type{\setupTABLE[column][1][alignmentcharacter={.},}\\
\type{aligncharacter=yes,align=middle]}\\
\type{\setupTABLE[column][2][alignmentcharacter={,},}\\
\type{aligncharacter=yes,align=middle]}\\
\type{\bTABLE}\\
\type{\bTR \bTH first \eTH \bTH second \eTH \bTH third \eTH \bTH fourth\eTH \eTR}\\
\type{\bTR \bTD 100.000,00\eTD \bTD 1,0 \eTD \bTD 100.000,00\eTD \bTD 1,0 \eTD \eTR}\\
\type{\bTR \bTD 10.000,00 \eTD \bTD 10,0 \eTD \bTD 10.000,00 \eTD \bTD 10,0 \eTD \eTR}\\
\type{\bTR \bTD 100,00 \eTD \bTD 1,00 \eTD \bTD 100,00 \eTD \bTD 1,00 \eTD \eTR}\\
\type{\bTR \bTD 10 \eTD \bTD 10,00 \eTD \bTD 10 \eTD \bTD 10,00 \eTD \eTR}\\
\type{\eTABLE}\\

\blank[3cm]

{
\setupTABLE[
	frame=off,
	]
\setupTABLE[column][first][
	leftframe=on,
	]
\setupTABLE[column][last][
	rightframe=on,
	]
\setupTABLE[row][first][
	topframe=on,
	]
\setupTABLE[row][first,last][
	bottomframe=on,
	]
\setupTABLE[column][1][
	alignmentcharacter={.},
	aligncharacter=yes,
	align=middle,
	]
\setupTABLE[column][2][
	alignmentcharacter={,},
	aligncharacter=yes,
	align=middle,
	]
\bTABLE
\bTR\bTH first \eTH\bTH second \eTH\bTH third \eTH\bTH fourth\eTH\eTR
\bTR\bTD 100.000,00\eTD\bTD 1,0 \eTD\bTD 100.000,00\eTD\bTD 1,0 \eTD\eTR
\bTR\bTD 10.000,00 \eTD\bTD 10,0 \eTD\bTD 10.000,00 \eTD\bTD 10,0 \eTD\eTR
\bTR\bTD 100,00 \eTD\bTD 1,00 \eTD\bTD 100,00 \eTD\bTD 1,00 \eTD\eTR
\bTR\bTD 10 \eTD\bTD 10,00 \eTD\bTD 10 \eTD\bTD 10,00 \eTD\eTR
\eTABLE
}

\Topic[mr]{Manuales y referencias}

\goto{Entrada principal}[url(https://wiki.contextgarden.net/TABLE)] de la wiki.

\blank

\goto{Manual}[url(http://dl.contextgarden.net/myway/NaturalTables.pdf)], por Willi Egger.

\blank

\goto{Ejemplos varios}[url(http://www.pragma-ade.com/general/manuals/enattab.pdf)], por Pragma-ade.




\stoptext