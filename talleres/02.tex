\usemodule[pre-01]
\mainlanguage[es]
\usemodule[chart]
%\showframe
\setupcolors[state=start]
\definelayer[LogoUTM]
\setupbodyfont[cmr,13pt]
\setuptype[color=blue]
\setuphead[subject][
		style=\tfa\bf
		]
\setuplayout[
		backspace=0.5cm,
		width=17.5cm,
		textwidth=16.25cm,
		]
\setupinteraction[
		state=start,
		color=middlecyan,
		style=\bf\ss
		]
\starttext
\StartTitlePage
  \CONTEXT ~~\&~~TeXworks\\  
{\tfxx
\goto{http://wiki.contextgarden.net}[url(http://wiki.contextgarden.net)]}
\blank
Taller 02
\blank
Referencias cruzadas,\\ imágenes y tablas
\blank[2cm]
\CONTEXT -ec\par   
{\externalfigure[figuras/utm_logo][width=2.5cm]}
\StopTitlePage 

\Topics{Contenido}

\Topic[refcru]{Referencias cruzadas}

Para poder utilizar referencias cruzadas a títulos, secciones, ecuaciones, figuras, tablas, entre otros, es necesario que el objeto a en cuestión tenga un nombre clave, definido de la siguiente forma:\blank
\type{\comando}{\bf [nombre clave]}\type{{Nombre real}}\blank
Ejemplos:\blank
\type{\chapter}{\bf [intro]}\type{{Introducción}}\blank
\type{\section}{\bf [refcru]}\type{{Referencias cruzadas}}\blank
\type{\placeformula}{\bf [pitagoras]}\\
\type{\startformula}\\
\type{a^2+b^2=c^2}\\
\type{\stopformula}\blank[2*big]
Para controlar el color y otros aspectos de las referencias cruzadas, hipervínculos, entre otros, haremos uso del comando de configuración:\blank
\type{\setupinteraction[state=start,color=name,style=name]}\page

\subject{Referencia numérica}
\type{\in{Texto previo}{Texto posterior}[Referencia]}\blank
{\bf Referencia} hace alusión al {\bf nombre clave} del objeto en cuestión.\\
{\bf Texto previo}~y~{\bf Texto posterior} son opcionales. Si son colocados también formarán parte del enlace.\\
El objeto a referenciar debe tener la característica de numerado. Para esto, al incluir títulos, utilizar de preferencia los comandos\type{\chapter},~\type{\section}~y derivados.\blank[line]
{\bf Ejemplos:}\\
...más información en la sección~\type{\in[refcru]}.\blank
da como resultado:\\
...más información en la sección~\in[refcru].\blank[2*line]
...más información en la~\type{\in{sección}{.}[refcru]}\blank
da como resultado:\\
...más información en la~\in{sección}{.}[refcru]\page

\subject{Referencia textual}
\type{\about[Referencia]}\blank
{\bf Ejemplo:}\\
...más información en la sección~\type{\about[refcru]}.\blank
da como resultado:\\
...más información en la sección~\about[refcru].\page

\subject{Referencia a página}
\type{\at{Texto previo}{Texto posterior}[Referencia]}\blank
{\bf Texto previo}~y~{\bf Texto posterior} son opcionales. Si son colocados también formarán parte del enlace.\blank
{\bf Ejemplos:}\\
...esta figura, ubicada en la página~\type{\at[refcru]}, es prueba que...\blank
da como resultado:\\
...esta figura, ubicada en la página~\at[refcru], es prueba que...\blank[3*line]
...esta figura, ubicada en la \type{\at{página~}[refcru]}, prueba que...\blank
da como resultado:\\
...esta figura, ubicada en la \at{página~}[refcru], prueba que...\page

\Topic{Imágenes}
\Topic{Tablas}
\stoptext
