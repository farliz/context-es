\usemodule[pre-01]
\mainlanguage[es]
\usemodule[chart]
%\showframe
\setupcolors[state=start]
\definelayer[LogoUTM]
\setupbodyfont[cmr,13pt]
\setuptype[color=blue]
\setuphead[subject][
		style=\tfa\bf
		]
\setuplayout[
		backspace=0.5cm,
		width=17.5cm,
		textwidth=16.25cm,
		]
\setupinteraction[
		state=start,
		color=middlecyan,
		style=\bf\ss
		]
\setupexternalfigures[directory={C:/Users/duonagem/Documents/GitHub/context-ec/talleres/figuras}]
\starttext
\StartTitlePage
  \CONTEXT ~~\&~~TeXworks\\  
{\tfxx
\goto{http://wiki.contextgarden.net}[url(http://wiki.contextgarden.net)]}
\blank
Taller 02
\blank
Referencias cruzadas,\\ imágenes y tablas
\blank[2cm]
\CONTEXT -ec\par   
{\externalfigure[figuras/utm_logo][width=2.5cm]}
\StopTitlePage 

\Topics{Contenido}

%%%%%%%%%%%%%%%%%%%%%%%%%%%%%%%%%%%%%%%%%%%%%%%%%%%%%%%%%%%%%%%%%%%%%%%%%%%%
\Topic[refcru]{Referencias cruzadas}

Para poder utilizar referencias cruzadas a títulos, secciones, ecuaciones, figuras, tablas, entre otros, es necesario que el objeto a en cuestión tenga un nombre clave, definido de la siguiente forma:\blank
\type{\comando}{\bf [nombre clave]}\type{{Nombre real}}\blank
Ejemplos:\blank
\type{\chapter}{\bf [intro]}\type{{Introducción}}\blank
\type{\section}{\bf [refcru]}\type{{Referencias cruzadas}}\blank
\type{\placeformula}{\bf [pitagoras]}\\
\type{\startformula}\\
\type{a^2+b^2=c^2}\\
\type{\stopformula}\blank[2*big]
Para controlar el color y otros aspectos de las referencias cruzadas se emplea el siguiente comando de configuración:\blank
\type{\setupinteraction[state=start,color=name,style=name,etc.]}\page

\subject{Referencia numérica}
\type{\in{Texto previo}{Texto posterior}[Referencia]}\blank
{\bf Referencia} hace alusión al {\bf nombre clave} del objeto en cuestión.\\
{\bf Texto previo}~y~{\bf Texto posterior} son opcionales. Si son colocados también formarán parte del enlace.\\
El objeto a referenciar debe tener la característica de numerado. Para esto, al incluir títulos, utilizar de preferencia los comandos\type{\chapter},~\type{\section}~y derivados.\blank[line]
{\bf Ejemplos:}\\
...más información en la sección~\type{\in[refcru]}.\blank
da como resultado:\\
...más información en la sección~\in[refcru].\blank[2*line]
...más información en la~\type{\in{sección}{.}[refcru]}\blank
da como resultado:\\
...más información en la~\in{sección}{.}[refcru]\page

\subject{Referencia textual}
\type{\about[Referencia]}\blank
{\bf Ejemplo:}\\
...más información en la sección~\type{\about[refcru]}.\blank
da como resultado:\\
...más información en la sección~\about[refcru].\page

\subject{Referencia a página}
\type{\at{Texto previo}{Texto posterior}[Referencia]}\blank
{\bf Texto previo}~y~{\bf Texto posterior} son opcionales. Si son colocados también formarán parte del enlace.\blank
{\bf Ejemplos:}\\
...esta figura, ubicada en la página~\type{\at[refcru]}, es prueba que...\blank
da como resultado:\\
...esta figura, ubicada en la página~\at[refcru], es prueba que...\blank[3*line]
...esta figura, ubicada en la \type{\at{página~}[refcru]}, prueba que...\blank
da como resultado:\\
...esta figura, ubicada en la \at{página~}[refcru], prueba que...\page

%%%%%%%%%%%%%%%%%%%%%%%%%%%%%%%%%%%%%%%%%%%%%%%%%%%%%%%%%%%%%%%%%%%%%
\Topic{Gráficos}

El comando más simple para insertar gráficos es:\blank
\type{\externalfigure[nombre.formato]}\blank
\CONTEXT~acepta los siguientes formatos de imagen:\blank
\startitemize[a, packed]
\item {\bf PDF}: .pdf
\item {\bf MPS}: .mps o .<digits> 
\item {\bf JPEG}: .jpg o .jpeg
\item {\bf PNG}: .png
\item {\bf JPEG2000}: .jp2
\item {\bf JBIG/JBIG2}: .jbig, .jbig2 o .jb2
\stopitemize

En muchos casos será necesario ajustar las dimensiones de los gráficos, para lo cual es necesario añadir parámetros:\blank
\type{\externalfigure[nombre.formato][height=dimensión,width=dimensión]}\blank
Se puede especificar el alto o el ancho y el programa reescalará la dimensión no definida. También es posible especificar ambas dimensiones a la vez o ninguna. En este último caso, la imagen será posicionada con su tamaño original.\page
{\bf Ejemplo:}\\
\type{\externalfigure[pinguinos.jpg][height=3cm,width=4cm]}\blank
da como resultado:\\
\externalfigure[pinguinos.jpg][height=3.5cm,width=4.5cm]\blank
Esta estructura de comando funciona sólo cuando el archivo se encuentra en el mismo directorio que el archivo .tex. Para insertar gráficos que no se encuentren en el mismo directorio será necesario especificar este último con el siguiente comando de configuración:\\
\type{\setupexternalfigures[directory={Ruta}]}\blank
{\bf Ruta} hace referencia a la dirección de la carpeta contenedora. Esta se debe escribir con barra inclinada hacia la derecha (/).\blank
{\bf Ejemplo de una ruta bien definida:}\\
C:/Users/duonagem/Documents/GitHub/context-ec/talleres/figuras\page

\subject{Objetos flotantes}
En trabajos académicos es prudente insertar gráficos como objetos flotantes. Esto significa que context insertará la imagen en el mejor lugar posible, según su algoritmo. También brinda ventajas como la numeración automática y, por ende, la capacidad de hacer referencias cruzadas.\blank
Para insertar un gráfico como objeto flotante se emplea el siguiente comando:\blank
\type{\placefigure[Posición][Referencia]{Pie de foto}...}\\
\type{...{\externalfigure[nombre.formato]}}\blank
{\bf Posición} se refiere a la alineación del gráfico, ya se izquierda, centro  o derecha. Si no se especifica, se insertará según la alineación por defecto o predefinida.\\
{\bf Referencia} es el nombre clave del gráfico para referencias cruzadas.\\
{\bf Pie de foto} es el texto que describirá el gráfico. Este es opcional.\page
{\bf Ejemplo:}\\
\type{\placefigure[right][ping1]{Pingüinos en la Antártida}...}\\
\type{...{\externalfigure[pinguinos.jpg]}}\blank
da como resultado (Nótese cómo se numera automáticamente la imagen):\\
\placefigure[middle][ping1]{Pingüinos en la Antártida}{\externalfigure[pinguinos.jpg]}\blank

%%%%%%%%%%%%%%%%%%%%%%%%%%%%%%%%%%%%%%%%%%%%%%%%%%%%%%%%%%%%%%%%%%
\Topic{Tablas}

\stoptext
