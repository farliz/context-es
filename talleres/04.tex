\usebtxdataset[references.bib]
\usebtxdefinitions[apa]
\setupbtx[apa:cite][etallimit=3,etaldisplay=1]


\usemodule[pre-01]
\mainlanguage[es]
\usemodule[chart]
%\showframe
\setupcolors[state=start]
\definelayer[LogoUTM]
\setupbodyfont[cmr,13pt]
\setuptype[color=blue]
\setuphead[subject][
		style=\bfb
		]
\setuplayout[
		backspace=0.5cm,
		width=17.5cm,
		textwidth=16.25cm,
		]
\setupinteraction[
		state=start,
		color=middlecyan,
		style=\bf\ss
		]
\def\BIBTEX{Bib\TEX}
\setupwhitespace[medium]

\starttext
\StartTitlePage
  \CONTEXT ~~\&~~TeXworks\\  
{\tfxx
\goto{http://wiki.contextgarden.net}[url(http://wiki.contextgarden.net)]}
\blank
Taller 04
\blank
Bibliografía
\blank[1cm]
\CONTEXT -ec

{
\setupTABLE[frame=off]
\bTABLE
\bTR
	\bTD[width=13.5cm]
{\externalfigure[figures/utm_logo][width=2.5cm]}
	\eTD
	\bTD
{\externalfigure[figures/context_logo][width=2.5cm]}
	\eTD
	\eTR
\eTABLE
}
\StopTitlePage 

\Topics{Contenido}

%%%%%%%%%%%%%%%%%%%%%%%%%%%%%%%%%%%%%%%%%%%%%%%%%%%%%%%%%%%%%%%%%%%%%%%%%%%%
\Topic[gen]{Generalidades}

\startitemize
	\item \CONTEXT~ adopta el formato \BIBTEX~ para gestionar la bibliografía. Información detallada acerca de este formato se puede encontrar en \goto{bibtex.org}[url(http://www.bibtex.org/)]
	\item La base de datos bibliográfica se almacena en un archivo de extensión {\bf .bib}, a manera de entradas.
	\item Si un documento incluye referencias, debe especificarse el archivo contenedor de la base de datos, así como otros detalles de estilo mediante los siguientes comandos:
\stopitemize

{\tfa
\type{\usebtxdataset[referencias.bib]}\\
\type{\usebtxdefinitions[apa]}\\
\type{\setupbtx[apa:cite][etallimit=2, etaldisplay=1]}\\
}

\startitemize
	\item Para insertar la lista de publicaciones citadas se emplea el siguiente comando:
\stopitemize

{\tfa
\type{\placelistofpublications}
}

{\externalfigure[figures/bib_file][width=15cm]}

\Topic[entradas]{Entradas bibliográficas}

\startitemize
	\item Una entrada bibliográfica se compone de varios elementos, además de una estructura específica: especificación del tipo de documento, identificador y variables. 
	\item La sintaxis de las entradas es la siguiente:
\stopitemize

\blank[big]

{\tta
@especificación\{identificador,\\
	variable1 = \{detalle\},\\
	variable2 = \{detalle\},\\
	variable3 = \{detalle\},\\
\}}

\page

\subject{Artículo científico}

{\tta
@article\{rodriguez2017,\\
doi     = \{10.33936/riemat.v2i2.1143\},\\
year    = \{2017\},\\
volume  = \{2\},\\
number  = \{2\},\\
pages   = \{1--5\},\\
author  = \{Rodríguez, M. and Vázquez, A. and Saltos, W.M. and Ramos, J.\},\\
title   = \{El potencial solar y la generación distribuida en la provincia de Manabí en el Ecuador\},\\
journal = \{Revista de Investigaciones en Energía, Medio Ambiente y Tecnología\},\\
\}}

\page

\subject{Libro}

{\tta
@book\{arias2003,\\
	title     = \{Periferias y nueva ciudad: el problema del paisaje en los procesos de dispersión urbana\},\\
	edition   = \{1\},\\
	author    = \{Arias, P.\},\\
	year      = \{2003\},\\
	publisher = \{Universidad de Sevilla\},\\
	address   = \{Sevilla\},\\
\}}


\page

\subject{Colección (Un libro en el cual cada capítulo tiene sus propios autores)}

{\tta
@incollection\{buroz1987,\\
	title     = \{Estudios Ambientales: Revisión de conceptos, metodología y presentación de resultados\},\\
	booktitle = \{Planificación Ambiental, una visión de conjunto\},\\
	edition   = \{1\},\\
	author    = \{Buroz, E.\},\\
	year      = \{1987\},
	publisher = \{Universidad Simón Bolívar and LAGOVEN, filial de Petróleos de Venezuela\},\\
	address   = \{Venezuela\}\\
\}
}

\page

\subject{Tesis}

{\tta
@mastersthesis\{corral2015,\\
  title  = \{Influencia del uso de suelo en el aprovechamiento de recursos naturales de la micro-cuenca del río Carrizal. Caso Julián y Severino\},\\
  author = \{Corral, G.M.\},\\
  school = \{ESPAM MFL\}\\
  year   = \{2015\}\\
\}\\

\blank

@phdthesis\{santos2017,\\
	title  = \{Metodología para el cálculo del límite de potencia eólica (LPE) en sistemas eléctricos débiles y distribuidos\},\\
	author = \{Santos, A.\},\\
	school = \{Universidad Tecnológica de La Habana\},\\
	year   = \{2017\}\\
\}\\
}

\page

\subject{Reporte técnico}

{\tta
@techreport\{MORA2003,\\
	title = \{Perturbaciones en la onda de tensión: Huecos [sag] y Sobretensiones [swell]\},\\
	author = \{Mora, J.J.\},\\
	institution = \{Universidad de Girona\},\\
	year = \{2003\},\\
	address = \{Girona\},\\
	url = \{http://eia.udg.es/~secse/curso_calidad/curso4-huecosdetension.pdf\}\\
\}\\
}

\page

\subject{Misceláneos (sitios web, manuales, periódicos, entre otros)}

{\tta
@misc\{guzman2018,\\
	author = \{Guzmán, J.\},\\
	year   = \{2018\},\\
	title  = \{¿Cómo se transforma el bagazo de caña de azúcar a energía eléctrica?. elsalvador.com\},\\
	url    = \{https://www.elsalvador.com/noticias/negocios/como-se-transforma-el-bagazo-de-cana-de-azucar-a-energia-electrica/439746/2018/\}\\
\}\\
}

\Topic[citas]{Citas}

\startitemize
	\item Las citas se realizan mediante el comando:
\stopitemize


{\tfb \type{\cite[tipo][identificador]}}

\blank

El tipo determina cómo se mostrará la cita en el texto. Dependiendo del contexto y el estilo de redacción se preferirá uno u otro. A continuación algunos ejemplos:

\blank

{
\setupTABLE[frame=off]
\setupTABLE[r][first,last][bottomframe=on]
\setupTABLE[r][first][style=\bfa]
\setupTABLE[c][even][align=middle,style=\tfa]
\setupTABLE[c][odd][align=middle,style=\tfa]
\bTABLE
\bTR \bTD[width=4cm] Tipo \eTD \bTD[width=4cm] Ejemplo       \eTD \eTR
\bTR \bTD authoryear   \eTD \bTD (Arias, 2003) \eTD \eTR
\bTR \bTD year         \eTD \bTD (2003)        \eTD \eTR
\bTR \bTD authoryears  \eTD \bTD Arias (2003)  \eTD \eTR
\eTABLE
}

\page

\subject{Ejemplos}

{\tfa
\type{El cambio climático es causado, en su mayoría, por ciertas actividades antrópicas \cite[authoryear][gomez2015].}
}

\blank

Da como resultado:

\blank

{\tfa
El cambio climático es causado, en su mayoría, por ciertas actividades antrópicas \cite[authoryear][gomez2015].
}


\page

{\tfa
\type{Según \cite[authoryears][gomez2015], el cambio climático es causado, en su mayoría, por ciertas actividades antrópicas.}
}

\blank

Da como resultado:

\blank

{\tfa
Según \cite[authoryears][gomez2015], el cambio climático es causado, en su mayoría, por ciertas actividades antrópicas.
}

\blank[2cm]

\subject{Bibliografía}
\placelistofpublications

\Topic[ref]{Referencias}

\goto{Manual}[url(https://modules.contextgarden.net/cgi-bin/module.cgi/action=find/name=bibman)] del módulo Bib, por Taco Hoekwater.

\blank

\goto{Entrada principal}[url(https://wiki.contextgarden.net/Bibliography)] de la Wiki.